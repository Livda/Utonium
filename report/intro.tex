\chapter*{Introduction}
\addcontentsline{toc}{chapter}{Introduction}

A l'heure actuelle, dans un monde où l'on cherche à aller toujours plus vite et à développer des applications sans connaissances en code, les générateurs sont devenus une alternative aux développeurs. Mais même pour les développeurs, générer du code est un moyen rapide et sûr d'avoir une base saine et normalement sans bug pour leur futures applications. Il existe de nombreux outils de générations (comme ceux dans la liste \cite{liste_generateurs}) et tous ne couvrent pas les mêmes champs. Certains servent à générer des squelettes de plus en plus poussés à partir de schémas UML, d'autre de permettre de générer du code \jv à partir de Python, etc.

Dans cette phase d'utilisation intensive de code généré, il est important de s'assurer que ce code est bel et bien fonctionnel. Le but étant de décharger le développeur d'écriture de certaines phases de développement, parfois lourde, complexes ou répétitives. Dans cet esprit d'optimisation du temps et de réduction des erreurs, des outils de vérifications et de contrôle sont nécessaires.

Une des solutions serait de faire des tests unitaires de toutes les fonctions générées dans le générateur. Un des inconvénients de cette méthode est son temps de mise en place beaucoup trop conséquent. Tester le code généré de cette manière reviendrait à perdre tous les bénéfices de la génération.

C'est dans cette optique que nous avons fait notre projet. Nous avons dû développer un outil de test automatique des générateurs de code.

\hspace{15pt}

%Description de la soution (tres succinte) & evaluation

Ce projet a pour but de développer un outil de test automatique des générateurs de code. Nous pouvions choisir de tester des générateurs proposés dans \cite{liste_generateurs}. Nous avons choisi d'utiliser Java et ses modules JUnit pour tester les générateurs suivants :

\todo{Fixer les générateurs}
\begin{itemize}
    \item javac,
    \item Bubbles (notre programme),
    \item moolinet \cite{moolinet}(un autre projet de test mené par Quentin Dufour et Loïck Bonniot)
\end{itemize}

Vous pouvez trouver le code de notre testeur sur le dépôt suivant :
https://gitlab.insa-rennes.fr/francois-boschet/utonium
