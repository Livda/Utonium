\chapter*{Conclusion}
\addcontentsline{toc}{chapter}{Conclusion}

Dans le cadre du cours de Test logiciel nous avons donc été amenés à développer
une application permettant de tester le code généré par de multiples générateurs.

Nous projet s'est construit autour de deux points centraux :

\begin{itemize}
    \item la facilité de génération des tests ;
    \item une compatibilité avec le plus de générateurs possible.
\end{itemize}

Nous avons donc réalisé un programme de tests de générateurs de code capable de compiler du code avec un compilateur fourni par l'utilisateur.
Ce code sera ensuite compilé et le résultat de l’exécution comparé avec le résultat attendu par l'utilisateur. Nous avons pris le partie de
comparer les résultats de l’exécution du code plutôt que le code généré lui-même, car cela offre une plus grande souplesse au développeur
du générateur et au développeur des tests. On suppose que si le code généré, produit le bon résultat une fois exécuté, alors le code est valide.
Le problème qui survient alors est comment tester suffisamment pour avoir le même niveau de certitude que l'analyse statique du code généré.

Au final notre solution est utilisable grâce à un .jar et à un fichier .json dont
la syntaxe est simplifiée au maximum. Java permet une utilisation sur tous les
systèmes d'exploitation et donc d'avoir une solution portable tout aussi simplement
que l'est son utilisation.
